%Estrutura "main" para crear documento con formato IEEE

\documentclass[conference]{IEEEtran}
\IEEEoverridecommandlockouts
% The preceding line is only needed to identify funding in the first footnote. If that is unneeded, please comment it out.\def
%\def\abstractname{Abstract}
\usepackage{cite}
\usepackage{amsmath,amssymb,amsfonts}
\usepackage{algorithmic}
\usepackage{graphicx}
\usepackage{textcomp}
\usepackage{xcolor}


\usepackage[utf8]{inputenc}
\def\BibTeX{{\rm B\kern-.05em{\sc i\kern-.025em b}\kern-.08em
    T\kern-.1667em\lower.7ex\hbox{E}\kern-.125emX}}

\begin{document}
    
    \title{Titulo}

    \author
    {
        \IEEEauthorblockN
        {
            Nombre1, Nombre2
        }
        \IEEEauthorblockA
        {
            \textit{Facultad}\\
            \textit{Universidad}\\
            ciudad, pais\\
            correo1\\correo2
        }
    }

    \maketitle

    \begin{abstract}
    resumen
    \end{abstract}

    \def\abstractname{abstract}
    \begin{abstract}
    abstract
    \end{abstract}

    \begin{IEEEkeywords}
    palabras clave
    \end{IEEEkeywords}

    \section{Introduccion}

        Inicio

    \section{desarollo}
        \subsection{parte1}

            Primera parte \cite{b1}

        \subsection{parte2}

            Segunda parte \cite{b2}
    
    \section{conclusion}

        final \cite{b1}\cite{b2}

    \begin{thebibliography}{00}
    \bibitem{b1} referencia1
    \bibitem{b2} referencia2
    \end{thebibliography}

\end{document}
