%Estrutura "main" para crear documento con formato IEEE, descargar junto con Lorem.tex

\documentclass[conference]{IEEEtran}
\IEEEoverridecommandlockouts
% The preceding line is only needed to identify funding in the first footnote. If that is unneeded, please comment it out.\def
%\def\abstractname{Abstract}
\usepackage{cite}
\usepackage{amsmath,amssymb,amsfonts}
\usepackage{algorithmic}
\usepackage{graphicx}
\usepackage{textcomp}
\usepackage{xcolor}


\usepackage[utf8]{inputenc}
\def\BibTeX{{\rm B\kern-.05em{\sc i\kern-.025em b}\kern-.08em
    T\kern-.1667em\lower.7ex\hbox{E}\kern-.125emX}}

\begin{document}
    
    \title{Titulo}

    \author
    {
        \IEEEauthorblockN
        {
            Nombre con Apellido1, Nombre con Apellido2
        }
        \IEEEauthorblockA
        {
            \textit{Facultad de Universidad Com\'un}\\
            \textit{Universidad Com\'un}\\
            ciudad, pais\\
            correo1\\correo2
        }
    }

    \maketitle

    \begin{abstract}
    resumen con mucho texto de relleno para parecer un "abstract" en espa\~nol, pero
    ya se sabr\'a que aqu\'i va el abstract en espa\~nol.
    \end{abstract}

    \def\abstractname{abstract}
    \begin{abstract}
    abstract con texto repetido de arriba para parecer un "abstract" en ingles, pero
    ya se sabr\'a que aqu\'i va el abstract en ingles.
    \end{abstract}

    \begin{IEEEkeywords}
    palabras clave, palabras clave necesarias, agregarlas exactas
    \end{IEEEkeywords}

    \section{Introduccion}

        Inicio, aqu\'i va la introducci\'on y mucho texto, este texto ser\'a agregado de otro archivo llamado \textbf{Lorem.tex}\\
        
        \input{Lorem.tex}

    \section{desarollo}
        \subsection{parte1}

            Primera parte, aqu\'i va alg\'un texto de relleno, se volver\'a a agregar el \textit{Lorem Ipsum} \cite{b1}
            
            \input{Lorem.tex}

        \subsection{parte2}

            Segunda parte, aqu\'i va alg\'un texto de relleno, se volver\'a a agregar el \textit{Lorem Ipsum} \cite{b2}
    
    \section{conclusion}

        final, aqu\'i va el final o la conclusi\'on de tu documento, como se puede observar se puede citar muchas veces con
        \textbf{\textit{\textbackslash cite\{nombre de etiqueta\}}} \cite{b1}\cite{b2} , tambien se agregar\'a
        el \textit{Lorem Ipsum}.
        
        \input{Lorem.tex}

    \begin{thebibliography}{00}
    \bibitem{b1} referencia1
    \bibitem{b2} referencia2
    \end{thebibliography}

\end{document}
